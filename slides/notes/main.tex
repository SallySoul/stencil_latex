\documentclass{beamer}
% Setup for bibliography
\usepackage[
backend=biber,
style=numeric-comp,
]{biblatex}
\addbibresource{../../references.bib}
\addbibresource{../../website_references.bib}


% Pretty self explanatory
% We sue this in title bits
\usepackage{datetime}

% Standard math packages / setup
\usepackage{amsmath} 
\usepackage{amsfonts}
\usepackage{amsthm}
\usepackage{amssymb} 
\usepackage{accents}
\usepackage{mathrsfs}
\usepackage{mathtools}

\usepackage{bm}

%\newtheorem{lemma}{Lemma}
%\newtheorem{theorem}{Theorem}
%\newtheorem{definition}{Definition}

% So we can import pngs
\usepackage{graphicx} 

% This gives us nice clickable links 
% https://www.overleaf.com/learn/latex/Hyperlinks#Styles_and_colours
\usepackage{hyperref}
\hypersetup{
    colorlinks=true,
    linkcolor=blue,
    citecolor=blue,
    filecolor=magenta,      
    urlcolor=cyan,
    pdftitle={Monte Carlo Methods (DRAFT)},
    pdfpagemode=FullScreen,
    }
\urlstyle{same}

% Allows us to define colors
% We use this in the next block, listings
\usepackage{color}
\definecolor{dkgreen}{rgb}{0,0.6,0}
\definecolor{gray}{rgb}{0.5,0.5,0.5}
\definecolor{mauve}{rgb}{0.58,0,0.82}

% Allows us to include 
\usepackage{listings}
\lstset{frame=tb,
  language={},
  aboveskip=3mm,
  belowskip=3mm,
  showstringspaces=false,
  columns=flexible,
  basicstyle={\small\ttfamily},
  numbers=none,
  numberstyle=\tiny\color{gray},
  keywordstyle=\color{blue},
  commentstyle=\color{dkgreen},
  stringstyle=\color{mauve},
  breaklines=true,
  breakatwhitespace=true,
  tabsize=4
}

% Adds bulletized outlines with outline environment
\usepackage{outlines}

% Tikz
\usepackage{tikz}

% Colors
\usepackage{xcolor}
\definecolor{uconnblue}{rgb}{0.08, 0.18, 0.28}
\definecolor{intactblue}{rgb}{0.13, 0.26, 0.45}
\definecolor{mastercamred}{rgb}{0.83, 0.01, 0.23}

% By default beamer slides are 4:3 , 128mm by 96mm
% Use to remove logo from some frames
% https://tex.stackexchange.com/questions/53781/how-can-i-include-the-logo-in-some-slides-and-remove-in-others-using-beamer/
\newif\ifplacelogo % create a new conditional
\placelogotrue % set it to true
\logo{\ifplacelogo\includegraphics[height=0.5cm]{../../assets/SBU_logos/horz_2clr_rgb_300ppi.png}\fi}

\usetheme{CambridgeUS}

%gets rid of footer
%will override 'frame number' instruction above
%comment out to revert to previous/default definitions
\setbeamertemplate{footline}{}

\AtBeginSection[]
{
  \begin{frame}
    \frametitle{Table of Contents}
    \tableofcontents[currentsection]
  \end{frame}
}

% Change bullets to triangles
% From https://tex.stackexchange.com/questions/11168/change-bullet-style-formatting-in-beamer
% Adapted from beamerinnerthemedfault.sty
\setbeamertemplate{itemize item}{\scriptsize\raise1.25pt\hbox{\donotcoloroutermaths$\blacktriangleright$}}
\setbeamertemplate{itemize subitem}{\tiny\raise1.5pt\hbox{\donotcoloroutermaths$\blacktriangleright$}}
\setbeamertemplate{itemize subsubitem}{\tiny\raise1.5pt\hbox{\donotcoloroutermaths$\blacktriangleright$}}
\setbeamertemplate{enumerate item}{\insertenumlabel.}
\setbeamertemplate{enumerate subitem}{\insertenumlabel.\insertsubenumlabel}
\setbeamertemplate{enumerate subsubitem}{\insertenumlabel.\insertsubenumlabel.\insertsubsubenumlabel}
\setbeamertemplate{enumerate mini template}{\insertenumlabel}

% Add footnote without mark
% https://tex.stackexchange.com/questions/30720/footnote-without-a-marker
\newcommand\blfootnote[1]{%
  \begingroup
  \renewcommand\thefootnote{}\footnote{#1}%
  \addtocounter{footnote}{-1}%
  \endgroup
}


% \shadowimage[width=8cm]{image}
%
% Provides a drop-shadow to images
%
% From
% https://tex.stackexchange.com/questions/81842/creating-a-drop-shadow-with-guassian-blur 
\usetikzlibrary{shadows,calc}

% code adapted from https://tex.stackexchange.com/a/11483/3954

% some parameters for customization
\def\shadowshift{3pt,-3pt}
\def\shadowradius{6pt}

\colorlet{innercolor}{black!60}
\colorlet{outercolor}{gray!05}

% this draws a shadow under a rectangle node
\newcommand\drawshadow[1]{
    \begin{pgfonlayer}{shadow}
        \shade[outercolor,inner color=innercolor,outer color=outercolor] ($(#1.south west)+(\shadowshift)+(\shadowradius/2,\shadowradius/2)$) circle (\shadowradius);
        \shade[outercolor,inner color=innercolor,outer color=outercolor] ($(#1.north west)+(\shadowshift)+(\shadowradius/2,-\shadowradius/2)$) circle (\shadowradius);
        \shade[outercolor,inner color=innercolor,outer color=outercolor] ($(#1.south east)+(\shadowshift)+(-\shadowradius/2,\shadowradius/2)$) circle (\shadowradius);
        \shade[outercolor,inner color=innercolor,outer color=outercolor] ($(#1.north east)+(\shadowshift)+(-\shadowradius/2,-\shadowradius/2)$) circle (\shadowradius);
        \shade[top color=innercolor,bottom color=outercolor] ($(#1.south west)+(\shadowshift)+(\shadowradius/2,-\shadowradius/2)$) rectangle ($(#1.south east)+(\shadowshift)+(-\shadowradius/2,\shadowradius/2)$);
        \shade[left color=innercolor,right color=outercolor] ($(#1.south east)+(\shadowshift)+(-\shadowradius/2,\shadowradius/2)$) rectangle ($(#1.north east)+(\shadowshift)+(\shadowradius/2,-\shadowradius/2)$);
        \shade[bottom color=innercolor,top color=outercolor] ($(#1.north west)+(\shadowshift)+(\shadowradius/2,-\shadowradius/2)$) rectangle ($(#1.north east)+(\shadowshift)+(-\shadowradius/2,\shadowradius/2)$);
        \shade[outercolor,right color=innercolor,left color=outercolor] ($(#1.south west)+(\shadowshift)+(-\shadowradius/2,\shadowradius/2)$) rectangle ($(#1.north west)+(\shadowshift)+(\shadowradius/2,-\shadowradius/2)$);
        \filldraw ($(#1.south west)+(\shadowshift)+(\shadowradius/2,\shadowradius/2)$) rectangle ($(#1.north east)+(\shadowshift)-(\shadowradius/2,\shadowradius/2)$);
    \end{pgfonlayer}
}

% create a shadow layer, so that we don't need to worry about overdrawing other things
\pgfdeclarelayer{shadow} 
\pgfsetlayers{shadow,main}


\newcommand\shadowimage[2][]{%
\begin{tikzpicture}
\node[anchor=south west,inner sep=0] (image) at (0,0) {\includegraphics[#1]{#2}};
\drawshadow{image}
\end{tikzpicture}}



%Information to be included in the title page:
\title{Stencil Notes}
\author{Russell Bentley}
\institute{Stony Brook}
\date{2024}

\begin{document}

\frame{\titlepage}

\section{LBM}
\begin{frame}{Nicolas Maquignon}
  \begin{outline}
    \1 Found through guest article on 
    \href{https://feaforall.com/implementation-lattice-boltzmann-method-lbm/}{FEA for all}

    \1 Published several articles about LBM \cite{Maquignon2014}, \cite{Maquignon2022}

    \2 \cite{Maquignon2014} extends model from another paper I can't find.
      
    \2 Condensation models seems like a good periodic problem.

    \1 Was on this article, which seems to have a good overview of LBM and related articles. 
    About splitting problem across GPUs.
    Has references about data access patterns on GPU
    \cite{Duchateau2015}
  \end{outline}
\end{frame}

\begin{frame}
  \begin{outline}
    \1 Attention, intention, and will in quantum physics \cite{Stapp1999}
  \end{outline}
\end{frame}

\begin{frame}{other stencil work}
  \begin{outline}
    \1 Tubostencil, a-periodic that trades space for runtime \cite{Liu2023turbo}
    \2 walBERlA, framwork for HPC block structured problems \cite{Bauer2021}
  \end{outline}
\end{frame}

\begin{frame}{LBM Theory}
  \begin{outline}
    \1 Cumulant paper, Hennig also cited for chimara transforms \cite{Geier2015}
  \end{outline}
\end{frame}

\begin{frame}{LBM Code Generation}
  \begin{outline}
    \1 New Code gen paper with lots of good stuff, \cite{Suffa2024}
    \1 Pystencil paper \cite{Bauer2019}
    \1 Kernkraft paper \cite{Hammer2017}
  \end{outline}
\end{frame}

\begin{frame}{Voronoi Mesh Stuff}
  \begin{outline}
    \1 \cite{Gaburro2021High} Young researchers paper on Higher order ale schemes and space-time slivers, considers how the mesh changes over time, very cool. Seems real paper is:

    \1 \cite{Gaburro2020High} Has springel on it! This one has all the math in it. Not in 3D yet though, 2D only. Plan to integrate with AREPO soon.

    \2 Point to \cite{Re2017Interpolation} for next steps on 3D

    \2 They are excited to try and model the hyperbolic formulation of continuum mechanics in \cite{Dumbser2016High}, and older paper on hyperbolic conservation laws \cite{Dumbser2011Universal}. And another hyperbolic model on viscous newtonian flows \cite{Peshkov2016Hyperbolic}.

    \2 \cite{Guermond2018Second} something about domain presercation in hyperbolic systems?

  \1 \cite{Gaburro2021Well} Gaburro presents a hyperbolic formulation and solver for general relativirty field equations, pretty cool

  \1 \cite{Dumbser2018Conformal} introduces FO-CCZ4 formulation of the einstein equations and presents an implementation. Hyperbolic!

    
  \end{outline}
\end{frame}

\begin{frame}{Vor Mesh More}
\begin{outline}
  \1 Great reference on ENO / WENO schemes \cite{zhang2016eno}
  \1 Original Arepo paper \cite{Springel2010Pur}
  \1 That didn't include MHD, that was added in \cite{pakmor2011mhd}
  \1 Dumbser 2008 paper with their ADER scheme \cite{dumbser2008unified} on unstructured meshes
  \1 Higher order finite volume framework in fortran \cite{antoniadis2022ucns3d}
  \1 \textsc{Gadget-4} paper \cite{springel2021simulating}, focus on scalability and mixed parallelism models. OOh and integer coords, eh mixed parallelism wasn't what I was expecting, openmp not a good fit, yadda yadda. BUT use of RMA and such IS interesting.
\end{outline}
\end{frame}

\begin{frame}[allowframebreaks]{References}
    \tiny
    \printbibliography
\end{frame}


\end{document}

