\section{Continuous Fourier Transforms}
Though we are interested in the discrete formulations of these transforms,
the discrete and continuous formulations share 
many properties and intuitions.
This material is largely from \cite{Damelin2012}.

\begin{definition}
  The Fourier transform, denoted as $\mathcal{F})$,
  takes a real valued function $f: \mathbb{R} \rightarrow \mathbb{R}$,
  and returns a new real valued function.
  $$
  \mathcal{F}[f](\lambda) := \frac{1}{\sqrt{2 \pi}} \int_{-\infty}^{\infty} f(t) e^{-i \lambda t} \ dt
  $$
\end{definition}

\begin{definition}
  Let $f$ and $g$ be $\mathbb{R} \rightarrow \mathbb{R}$ functions. 
  We say the convolution of $f$ and $g$, 
  denoted as $f * g$, is defined to be
  $$
  (f * g)(t) = \int_{-\infty}^{\infty}f(t - x)g(x) \ dx
  $$
\end{definition}

\begin{lemma}
  $$
  \int_{-\infty}^{\infty} |f * g(t)| \ dt = \int_{-\infty}^{\infty} |f(x)| \ dx \int_{-\infty}^{\infty} | g(t) | dt
  $$
\end{lemma}

\begin{theorem}
$$
\hat{h}(\lambda) = \hat{f}(\lambda) \hat{f}(\lambda)
$$
\end{theorem}

\begin{theorem}
Test
\end{theorem}
