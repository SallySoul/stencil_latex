\documentclass{article}
% Setup for bibliography
\usepackage[
backend=biber,
style=numeric-comp,
]{biblatex}
\addbibresource{../../references.bib}
\addbibresource{../../website_references.bib}


% Pretty self explanatory
% We sue this in title bits
\usepackage{datetime}

% Standard math packages / setup
\usepackage{amsmath} 
\usepackage{amsfonts}
\usepackage{amsthm}
\usepackage{amssymb} 
\usepackage{accents}
\usepackage{mathrsfs}
\usepackage{mathtools}

\usepackage{bm}

%\newtheorem{lemma}{Lemma}
%\newtheorem{theorem}{Theorem}
%\newtheorem{definition}{Definition}

% So we can import pngs
\usepackage{graphicx} 

% This gives us nice clickable links 
% https://www.overleaf.com/learn/latex/Hyperlinks#Styles_and_colours
\usepackage{hyperref}
\hypersetup{
    colorlinks=true,
    linkcolor=blue,
    citecolor=blue,
    filecolor=magenta,      
    urlcolor=cyan,
    pdftitle={Monte Carlo Methods (DRAFT)},
    pdfpagemode=FullScreen,
    }
\urlstyle{same}

% Allows us to define colors
% We use this in the next block, listings
\usepackage{color}
\definecolor{dkgreen}{rgb}{0,0.6,0}
\definecolor{gray}{rgb}{0.5,0.5,0.5}
\definecolor{mauve}{rgb}{0.58,0,0.82}

% Allows us to include 
\usepackage{listings}
\lstset{frame=tb,
  language={},
  aboveskip=3mm,
  belowskip=3mm,
  showstringspaces=false,
  columns=flexible,
  basicstyle={\small\ttfamily},
  numbers=none,
  numberstyle=\tiny\color{gray},
  keywordstyle=\color{blue},
  commentstyle=\color{dkgreen},
  stringstyle=\color{mauve},
  breaklines=true,
  breakatwhitespace=true,
  tabsize=4
}

% Adds bulletized outlines with outline environment
\usepackage{outlines}

% Tikz
\usepackage{tikz}

% Colors
\usepackage{xcolor}
\definecolor{uconnblue}{rgb}{0.08, 0.18, 0.28}
\definecolor{intactblue}{rgb}{0.13, 0.26, 0.45}
\definecolor{mastercamred}{rgb}{0.83, 0.01, 0.23}


\usepackage[
  letterpaper,
  left=2cm,
  right=2cm,
  top=2cm,
  bottom=2cm
]{geometry}

\setlength{\parindent}{20pt}

\title{CSE 528 Final Project Proposal}
\author{Russell Bentley}

\begin{document}

\twocolumn

\maketitle
\section{Introduction}

Kinetic methods for solving fluid flow,
like the \textit{Lattice Boltzman method}(LBM) offer several advantages
over traditional methods like finite volume or smooth particle hydrodynamics.
In particular, they are highly parallelizable 
and by construction solve for compressible fluids which negates the need 
for a global pressure solve. 
LBM works by modelling the probability distribution 
of particles traveling between nodes on a fixed set of discretized velocities.
LBM does a single time step with two operators.
First a streaming operator moves the particle distributions along the lattice.
Then a collision operator relaxes the velocities at a given node towards 
an equilibrium state.

Some recent advances in LBM have focused on increasing accuracy, 
an issue with traditional LBM implementations. 
In particular, a lack of \textit{Galilean invariance} 
was identified as a critical issue \cite{Geier2006}.
That is, traditional collision operators
like BGK will have different behavior depending on the bulk velocity across a given node.
BGK in particular relaxes all the moments of velocity at a uniform rate.
The central-moment method \cite{Geier2006, De2017, De2019, Li2020} seeks to fix this by
both finding moments of velocity relative to the bulk velocity and relaxing 
these moments at independent rates.

Two recent papers in particular have advanced the use of LBM for accurate 
treatment of fluids in Computer Graphics \cite{Li2020, Lyu2021}.
These two papers collectively describe how to find appropriate relaxation rates,
handle conversion from real units to LBM units, and an accurate treatment of fluid-solid coupling.

\section{Proposed Work}
My proposal is to implement an LBM solver framework 
based on the central-moment principle, as well as implement fluid-solid interation.
I do not intend to handle full fluid-solid coupling as described in \cite{Lyu2021},
but I would like to immerse fixed geometry into the fluid flow.

This project is composed of two components, 
the solver framework and the rendering pipeline.
My aim is to spend most of my time on the solver framework,
and largely use off the shelf tools for 
rendering.

The solver will largely run on the GPU.
While solvers of this nature are typically implemented in CUDA,
my goal is to target \href{https://wgpu.rs}{wgpu} instead.
Wgpu is a rust library that implements the \href{https://developer.mozilla.org/en-US/docs/Web/API/WebGPU_API}{WebGPU API}.
Here's an outline of why I want to do it this way:
\begin{outline}
  \1 I would like to write the host side in Rust, which I prefer to Cpp / C.
  \1 This implementation would be more portable.
    \2 I want to support Apple GPUs, which often have access to more memory than NVIDIA GPUs.
    \2 I would like to support GPUS from AMD.
    \2 Future versions could be deployed to web-browsers.
  \1 This would be trying something new.
\end{outline}

Generating high quality images is an important part of the project.
My initial idea is to utilize \href{https://www.openvdb.org}{OpenVDB}
as an intermediate form for the volumetric data the solver will generate.
From there, I would like to use \href{https://www.blender.org}{Blender}
to create volumetric renders.
This pipeline will be partially automated,
but will also involve some per-scene manual setup.

My backup option for rendering is to use \href{https://vtk.org}{VTK} and my intermediate format,
and to render with \href{https://www.paraview.org}{Paraview}.
For contour and vector plots this may be my best option, so it will likely be supported
regardless.

\section{Proposed Timeline}

\begin{outline}
  \1 Demonstrate VDB based volumentric Rendering
    \2 Within next $2$ weeks.
  \1 2D Fluid Solver no-solids
    \2 Within the $4$ weeks. 
  \1 3D Fluid solver no-solids
    \2 Within the next $5$ weeks.
  \1 Random Surface Sampling
    \2 Within the next $6$ weeks.
  \1 Fluid-Solid coupling
    \2 Within the next $7$ weeks.
  \1 Fluid-solid rendering
    \2 With final Deliverable
\end{outline}

\printbibliography

\end{document}

