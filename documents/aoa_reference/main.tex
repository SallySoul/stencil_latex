\documentclass[10pt]{article}
% Setup for bibliography
\usepackage[
backend=biber,
style=numeric-comp,
]{biblatex}
\addbibresource{../../references.bib}
\addbibresource{../../website_references.bib}


% Pretty self explanatory
% We sue this in title bits
\usepackage{datetime}

% Standard math packages / setup
\usepackage{amsmath} 
\usepackage{amsfonts}
\usepackage{amsthm}
\usepackage{amssymb} 
\usepackage{accents}
\usepackage{mathrsfs}
\usepackage{mathtools}

\usepackage{bm}

%\newtheorem{lemma}{Lemma}
%\newtheorem{theorem}{Theorem}
%\newtheorem{definition}{Definition}

% So we can import pngs
\usepackage{graphicx} 

% This gives us nice clickable links 
% https://www.overleaf.com/learn/latex/Hyperlinks#Styles_and_colours
\usepackage{hyperref}
\hypersetup{
    colorlinks=true,
    linkcolor=blue,
    citecolor=blue,
    filecolor=magenta,      
    urlcolor=cyan,
    pdftitle={Monte Carlo Methods (DRAFT)},
    pdfpagemode=FullScreen,
    }
\urlstyle{same}

% Allows us to define colors
% We use this in the next block, listings
\usepackage{color}
\definecolor{dkgreen}{rgb}{0,0.6,0}
\definecolor{gray}{rgb}{0.5,0.5,0.5}
\definecolor{mauve}{rgb}{0.58,0,0.82}

% Allows us to include 
\usepackage{listings}
\lstset{frame=tb,
  language={},
  aboveskip=3mm,
  belowskip=3mm,
  showstringspaces=false,
  columns=flexible,
  basicstyle={\small\ttfamily},
  numbers=none,
  numberstyle=\tiny\color{gray},
  keywordstyle=\color{blue},
  commentstyle=\color{dkgreen},
  stringstyle=\color{mauve},
  breaklines=true,
  breakatwhitespace=true,
  tabsize=4
}

% Adds bulletized outlines with outline environment
\usepackage{outlines}

% Tikz
\usepackage{tikz}

% Colors
\usepackage{xcolor}
\definecolor{uconnblue}{rgb}{0.08, 0.18, 0.28}
\definecolor{intactblue}{rgb}{0.13, 0.26, 0.45}
\definecolor{mastercamred}{rgb}{0.83, 0.01, 0.23}


\usepackage[
  letterpaper,
  left=2cm,
  right=2cm,
  top=2cm,
  bottom=2cm
]{geometry}

\setlength{\parindent}{20pt}

\title{Analysis of Algorithms Midterm 1 Reference}
\author{Russell Bentley}

\begin{document}

\twocolumn

\section{Asymptotic Notation}
Definitions, abbreviated \footnote{\url{https://web.mit.edu/broder/Public/asymptotics-cheatsheet.pdf}} for this class
\hrule
$f = \Theta(g)$ \\
$f$ grows at the same rate as $g$ \\
$\exists \ n_0$ and $c_1, c_2 > 0$ s.t. \\
$\forall \ n > n_0, c_1 g(n) \leq |f(n)| \leq c_2 g(n)$
\hrule
$f = O(g)$ \\
$f$ grows no faster than $g$ \\
$\exists \ n_0$ and $c > 0$ s.t. \\
$\forall \ n > n_0, |f(n)| \leq cg(n)$
\hrule
$f = \Omega(g)$ \\ 
$f$ grows at least as fast as $g$ \\
$\exists \ n_0$ and $c > 0$ s.t. \\
$\forall \ n > n_0, c g(n) \leq |f(n)|$
\hrule
$f = o(g)$ \\ 
$f$ grows no faster than $g$ \\
$\forall c>0, \exists \ n_0$ s.t. \\
$\forall \ n > n_0, |f(n)| \leq c g(n)$
\hrule

Asymptotic Relationships
\begin{align*}
  f = O(g), f = \Omega(g) &\iff f = \Theta(g) \\
  f = O(g) &\iff g = \Omega(f) \\
  f = o(g) &\implies f = O(g) \\
  \log n = O(n^\epsilon) &\ \forall \ \epsilon > 0 \\
\end{align*}

\section{Geometric Series}
Series where each successive term is a constant ratio of the prior.
$$
S_n = \sum_{i = 0}^n ar^i
$$
Useful things:
\begin{itemize}
  \item When $|r| < 1$ we have $\sum_{i = 0}^{\infty} ar^i = \frac{a}{1 - r}$
  \item For $|r| > 1$, $\sum_{i=0}^n f(n) r^i = O(f(n))$
  \item $1 + \frac{1}{2} + \frac{1}{4} + \cdots = 2$
  \item When $r > 1$, we have $S_n = \sum_{i=0}^n ar^i \approx a r^n$
\end{itemize}

\section{Logarithm Rules}

\begin{align*}
  \log_a 1 &= 0 \\
  \log_a a &= 1 \\
  \log_a a^n &= n \\
  \log_a (xy) &= \log_a x + \log b \\
  \log_a (x / y) &= \log_a x - \log_a y \\
  \log_a x^n &= n \log_a x \\
  \log_a x &= \log_b x / \log_b a \\
\end{align*}

Harmonic number equivalence.
$$
\sum_{i = 1}^n \frac{1}{i} = H_n \approx \ln n
$$

Derivative \& Integrals
$$
\frac{\partial}{\partial n} \ln n = \frac{1}{n}
$$

\section{Master Theorem}
Given a recurrence 
$$T(n) = \left\{
  \begin{array}{l l}
    \Theta(1) & \text{if }n \leq 1 \\
    a\cdot T\left(\frac{n}{b}\right) + f(n) & \text{otherwise} \\
  \end{array}\right.$$
where
$a \geq 1$, $b > 1$, and $f(n)$ is asymptotically positive we can bound
the work with:


\begin{tabular}{l l}
1: $f(n)$ polynomially smaller than $n^{\log_b a}$ & $\implies \Theta(n^{\log_b a})$\\
2: $f(n)$ polynomially equal to $n^{\log_b a}$ & $\implies \Theta(f(n) \log n)$\\
3: $f(n)$ polynomially larger than $n^{\log_b a}$ & $\implies \Theta(f(n))$\\
\end{tabular}

To prove one of the three cases you need to show that: \\
1: $\exists \ \epsilon > 0$ s.t. $f(n) = O\left( n^{\log_b(a - \epsilon})\right)$ \\
2: $f(n) = \Theta\left( n^{\log_b a})\right)$ \\
3: $\exists \ \epsilon > 0$ s.t. $f(n) = \Omega\left( n^{\log_b(a + \epsilon})\right)$\\
and $\exists c \leq 1$ s.t. $af\left(\frac{n}{b}\right) \leq cf(n)$


\section{Algorithms}

\subsection{Matrix Multiplication}

$A \cdot B = C$ where
$$c_{ij} = \sum_{k = 1}^n a_{ik} b_{kj}$$
Naive is $O(n^3)$ \\
Strassen's Algorithm is $O\left(n^{\log_2 7}\right) = O\left( n^{2.81}\right)$

\section{Akra-Bazzi}

$$
T(x) = \left\{ \begin{array}{ll}
      \Theta(1), & \text{if } 1 \leq x \leq x_0 \\
      \sum_{i = 1}^k a_i T(b_i x) + g(x) & \text{if } x > x_0\\
\end{array}\right.
$$
where
\begin{itemize}
  \item $k \geq 1$, integer constant
  \item $a_i > 0$ are constants
  \item $0 < b_i < 1$ are constants
  \item $x \geq 1$ is a real number
  \item $x_0 \geq \text{max}\left\{\frac{1}{b_i}, \frac{1}{1 - b_i} \right\}$ over all $i$
  \item $g(x)$ is non-negative, satisfies polynomial growth condition
\end{itemize}
Let $p$ be the unique real number such that $\sum_{i = 1}^k a_i b_i^p = 1$
then
$$
T(x) = \Theta\left(
x^p \left(
  1 + \int_{1}^x \frac{g(u)}{u^{p + 1}} \ du
\right)
\right)
$$

\section{Generating Functions}
Encode a sequence (cost for different values of $n$, for example) as coefficients in a power series.
$$
s_0, s_1, s_2, \cdots \rightarrow S(z) = s_0 + s_1 z + s_2 z^2 + \cdots + s_i z^i + \cdots
$$

In the Fibonacci example we have $F(Z)$, where each $f_i$ coefficient is the $i^{th}$ Fibonacci number.
Abbreviating some steps:
\begin{align*}
  F(z) &= zF(z) + z^2 F(z) + z \\
  &= \frac{z}{1 - z - z^2} \\
  &= \frac{z}{(1 - \phi z) (1 + \hat{\phi} z)} \\
  &= \frac{1}{\sqrt{5}}\left( \frac{1}{1 - \phi z} - \frac{1}{1 - \hat{\phi}z}\right) \\
  &= \frac{1}{\sqrt{5}} \sum_{n} (\phi^n - \hat{\phi}^n)z^n \\
\end{align*}
Which gives us a closed form for each $f_i$.

\section{Amortized Analysis}

Analyse the cost of operations within the context of a sequence of operations.
I.e. if operation $B$ costs $\Theta(n)$, but only after $n$ $A$ operations that take $\Theta(1)$ time,
usually means that we can view $B$ as having $\Theta(1)$ amortized cost.

\subsection{Accounting Method}

Create and ``account'' for each of some entity in the data structure, and ensure the amount in those accounts is an invariant.
$$
\bigwedge\limits_{V_j \in D} \text{credit}(V_j) = c
$$
For some constant $c$, often $1$.

Then the amortized cost of some operation is
$$
\hat{c}_i = c_i + \delta_i
$$
Cost of change in credits to maintain invariant

\subsection{Potential Method}

Represent total ``credits'' in data structure at time $i$ with potential function $\Phi(D_i)$.
Ensure:
\begin{itemize}
  \item $\Phi(D_0) = 0$
  \item $\Phi(D_i) \geq 0 \ \forall i$
\end{itemize}

Then the amortized cost of an operation
$$
\hat{c}_i = c_i + \Phi(D_{i}) - \Phi(D_{i - 1})
$$
Where $c_i$ is the actual cost, plus the potential change.

\section{High Probability Bounds}

Chernoff Bounds should be in the back of the exam if needed. \\

A high probability bound goes to zero in $n$ or some other variable.
But usually $n$ is the goal, $\frac{1}{n^p}$ for $p > 1$.

In the case of proving high probability $\Theta$, we have to show that
$$
1 - \left(Pr[X \leq (1 - \delta) \mu] + Pr[X \geq (1 + \delta)\mu]\right)
$$
goes to $0$ in $n$


\end{document}


