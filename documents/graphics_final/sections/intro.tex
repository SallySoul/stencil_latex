\begin{figure*}

\end{figure*}

\section{Intoduction}

The lattice Boltzmann method (LBM) 
is a method for numerically modelling fluid dynamics
that benefits from mapping naturally onto GPGPU architectures.
However LBM based models have traditionally struggled to 
properly resolve turbulent flows.
Given that the visual interest of fluid dynamics is often 
derived from turbulence, this has limited the use of LBM 
for graphics applications.

Active research has rapidly evolved LBM based models,
and new collision operators like CM-MRT (see section \ref{sec:cm-mrt})
have greatly improve accuracy when modelling turbulent flows.
CM-MRT has seen adoption by graphics 
researchers \cite{Li2020, Li2024, Lyu2021} due
to its ability to better resolve visually interesting
fluid dynamics.

The proposal for my final project was to
implement a GPU based LBM simulation utilizing the CM-MRT collision 
operator.
I was able to implement most of the ideas from
my proposal.
My implementation is described in section 
\ref{sec:implementation}.
Two key ideas from the proposal that I was unable to tackle
were tracer particle based volumetric rendering
and more advanced fluid solid coupling, see section \ref{sec:futurework}.

My goals were to to gain experience with both GPU programming and 
LBM based modelling.
I succeeded on both of these counts.
I feel far more confident in my ability to utilize compute shaders,
with the added bonus of learning about computer algebra based
code generation.
My greatest success though is on the theory side.
I have gained a foothold into
the state of the art for LBM based methods and
the research being conducted in this area.
