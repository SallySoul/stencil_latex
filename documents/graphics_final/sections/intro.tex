\section{Intoduction}

The lattice Boltzmann method (LBM) 
is a method for numerically modelling fluid dynamics
that benefits from mapping naturally onto GPGPU architectures.
However LBM based models have traditionally struggled to 
properly resolve turbulent flows.
Given that the visual interest of fluid dynamics is often 
derived from turbulence, this has limited the use of LBM 
for graphics applications.

Active research has rapidly evolved LBM based models,
and new collision operators like CM-MRT (see section \ref{sec:cm-mrt})
greatly improved accuracy when modelling turbulent flows.
CM-MRT has seen adoption by graphics 
researchers \cite{Li2020, Li2024, Lyu2021} due
to its ability to better resolve visually interesting
fluid dynamics.

The proposal for my final project was to
implement a GPU based LBM simulation utilizing the CM-MRT collision 
operator in order to gain experience with both GPU programming and 
LBM models.
In the end I was unable to fully complete this goal, 
but I do not see it as a failure.
As evidenced by this paper and my project I was able to implement 
a simulation using LBM to generate pictures and came close, I believe,
to a working CM-MRT operator.
My goals of gaining experience with the material have been 
achieved.

An overview of my implementation and its 
components is presented in section \ref{sec:implementation}.
The theory and related work is explored in \ref{sec:relatedwork}.
Future work is discussed in \ref{sec:futurework}.
