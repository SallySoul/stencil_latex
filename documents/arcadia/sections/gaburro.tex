\section{Gaburro High Order Methods}

\subsection{Introduction}

The work of Elana Gaburro is a direction of interest, namely high order ADER-DG methods.
She has a useful \href{https://www.elenagaburro.it/}{personal site}.

\subsection{Notes}

Mostly reading \cite{Gaburro2020High} at the moment.

CFL stability condition
\begin{align}
  \Delta t &< \text{CFL}\left(
    \frac{|P_i^n|}{(2N + 1) | \lambda_{\text{max},i}| \sum_{\nabla P_{i_j}^n} |\ell_{i_j}|}
  \right), \quad \forall P^{n}_i \in \Omega^n
\end{align}

Modal space-time basis of the polynomails of degree $M$ in $d + 1$ dimensions (space and time).
\begin{align}
  \theta_{\ell}(x, y, t)|_{C^n_i} &= 
  \frac{(x - x^n_{b_i})^{p_{\ell}}}{p_{\ell}!h_i^{p_{\ell}}}
\end{align}

\subsection{Recent Papers}

\begin{outline}
  \1 (2024) ``Very high order treatment of embedded curved boundaries in compressible flows: ADER discontinuous Galerkin with a space-time Reconstruction for Off-site data'' \cite{ciallella2024very}
  \2 High order immersed boundary conditions
  \2 Discusses recent reconstruction for off-site data (ROD), recent technique for finite Volumes
  \2 I think, they extend this to ADER-DG stuff?

  \1 (2024) ``A well-balanced discontinuous Galerkin method for the first--order Z4 formulation of the Einstein--Euler system'' \cite{dumbser2024well}
  \2 Novel first order reformation of Einstein equations (Z4), show that its hyperbolic
  \2 Employ Higher order Discntinuous Galerkin (DG) numerical scheme with adaptive mesh refinement (AMR), local time steping, and a posteriori sub cell finite volume limiter
  \2 Well balanced preserve a priori known equilibriums over long time periods

  \1 (2023) ``High order entropy preserving ADER-DG schemes'' \cite{gaburro2023high}
  \2 Good resrouce on ADER-DG schemes as expected
  \2 Adds an additiona constrain on entropy, cites an older paper \cite{harten1982symmetric} 

  \1 (2022) ``Continuous finite element subgrid basis functions for discontinuous Galerkin schemes on unstructured polygonal Voronoi meshes'' \cite{boscheri2022continuous}
  \2 Something about using sub-triangles of voronoi element to create finite element basis,
  \2 Does their usual space-time ADER thing, but no topology changes

  \1 (2022) ``A well-balanced discontinuous Galerkin method for the first--order Z4 formulation of the Einstein--Euler system'' \cite{Gaburro2021High}
  \2 \cite{Gaburro2020High} as a book chapter?


\end{outline}

\subsection{Related Papers}

\begin{outline}
  \1 (1983) ``On the symmetric form of systems of conservation laws with entropy'' \cite{harten1983symmetric}
  \2 Math paper, didn't read deeply
  \2 Hyperbolic systems, shows how to derive an entropy function for some of them? Something about symmetry? Uses matrix represenations
  \2 Doesn't seem to provide much context for what they mean by entropy
\end{outline}


