\usepackage{amsmath,amssymb,amsthm, bm}
\usepackage{graphicx,xcolor}
\usepackage[mathscr]{euscript}
\usepackage{tikz}
\usetikzlibrary{trees,arrows,automata}
\usetikzlibrary{shapes.geometric} % for oval nodes
\usetikzlibrary{shadows,fadings}
\usepackage{wrapfig}   % for wrapfig function
\usepackage{stmaryrd} % for llceil function
\usepackage{url}
\usepackage{pifont}
\usepackage{lipsum}
\usetikzlibrary{arrows,shapes,positioning,shadows,trees}
\usepackage{multirow}
\usepackage{mathtools}
%---- TCOLORBOX -----
\usepackage{tcolorbox}
\tcbuselibrary{skins}
%----- TABLE COLOR -----
\usepackage{environ}
\usepackage{colortbl}
\usepackage{hhline}
\usepackage{rotating}
\usepackage{array}
\usepackage{libertine}
\usepackage{enumitem}
\usepackage{mdframed}
\usepackage{cancel}
\usepackage{soul}
\usepackage{fontawesome}
\usepackage{physics}
\usepackage{xspace}
\usepackage{hhline}
\usepackage{pgfplots}
\pgfplotsset{compat=1.17}
\usepackage{tikz}
\usepackage{tikz-3dplot}
\usepackage{makecell}
\usepackage{arydshln}
\setlength{\dashlinedash}{2pt}  % Dot size (default is 4pt)
\setlength{\dashlinegap}{0.5pt}     % Gap between dots (default is 2pt)
\setlength{\arrayrulewidth}{0.5pt} % Thickness of the line


\usepackage{etoolbox}

\newbool{draft_formatting}
\setbool{draft_formatting}{false} % use to change formatting to the draft version

\newbool{arxiv_version}
\setbool{arxiv_version}{false} % use to change formatting to the Arxiv version

%\usepackage[dvipsnames,table]{xcolor}
%\usepackage{mdframed}
%\usepackage{amsfonts}
%\usepackage[mathscr]{euscript}
%\usepackage{tikz}
%\usetikzlibrary{trees,arrows,automata}
%\usetikzlibrary{shapes.geometric} % for oval nodes
%\usetikzlibrary{shadows,fadings}
%\usepackage{wrapfig}   % for wrapfig function
%\usepackage{stmaryrd} % for llceil function
%\usepackage{url}
%\usepackage{pifont}
%\usetikzlibrary{arrows,shapes,positioning,shadows,trees}
%---- TCOLORBOX -----
%\usepackage{tcolorbox}
%\tcbuselibrary{skins}
%----- TABLE COLOR -----
%\usepackage{environ}
%\usepackage{colortbl}

%\usepackage{draftwatermark}
%\SetWatermarkLightness{ 0.9 }
%\SetWatermarkText{DRAFT\\COPY}
%\SetWatermarkScale{ 0.3 }

\ifbool{draft_formatting}{
    \usepackage{xwatermark}
    \newwatermark[%
      allpages,% show on all pages
      angle=55,% rotate by 55 degrees
      scale=2.2,% scale by 2.75
      textcolor=gray!15
    ]{WORK IN PROGRESS\\PLEASE DO NOT SHARE}
}{}

\graphicspath{{SPAA-2021-Camera-Ready/images/}}

\newcommand{\Periodic}{\textsc{StencilFFT-P}}
\newcommand{\Aperiodic}{\textsc{StencilFFT-A}}
\newcommand{\Boundary}{\textsc{RecursiveBoundary}}

\newcommand{\StencilViaFFTPeriodic}{\textsc{PStencil-1D-FFT}\xspace}
\newcommand{\StencilViaFFTPeriodicdD}{\textsc{PStencil-$d$D-FFT}}
\newcommand{\EpsilonSliceStencilViaFFT}{\textsc{$\epsilon$-Slice-Stencil-FFT}}
\newcommand{\TwoWayStencilViaFFT}{\textsc{AStencil-1D-FFT}\xspace}
\newcommand{\TwoWayBoundaryViaFFT}{\textsc{ABoundary-1D-FFT}\xspace}
\newcommand{\TwoWayStencilViaFFTdD}{\textsc{AStencil-$d$D-FFT}\xspace}
\newcommand{\TwoWayBoundaryViaFFTdD}{\textsc{ABoundary-$d$D-FFT}\xspace}
\newcommand{\StencilviaNestedLoopPeriodic}{\textsc{PStencil-Loop}}
\newcommand{\StencilviaNestedLoopAperiodic}{\textsc{AStencil-Loop}}
\newcommand{\MFFT}{\textsc{Multi-FFT}}
\newcommand{\IMFFT}{\textsc{Inverse-Multi-FFT}}

\newcommand{\fft}{\mathcal{F}}
\newcommand{\ifft}{\mathcal{F}^{-1}}
\newcommand{\pluto}{PLuTo}
\newcommand{\plutostandard}{\texttt{standard}}
\newcommand{\plutodiamond}{\texttt{diamond}}

\newcommand{\hide}[1]{}
\newcommand{\Arxiv}[1]{
    \ifbool{arxiv_version}{#1}{}
}
%\newtheorem{property}{Property}
%\newtheorem{theorem}{Theorem}
%\newtheorem{definition}{Definition}
\newtheorem{assumption}{Assumption}
\newtheorem{rules}{Rule}
\newcommand{\floor}[1]      {\left\lfloor #1 \right\rfloor}
\newcommand{\ceil}[1]       {\left\lceil #1 \right\rceil}
\newcommand{\highlight}[1]{\textit{\textbf{#1}}}
\newcommand{\highlighttitle}[1]{\textbf{#1}}
\newcommand\pa[1]{\ensuremath{\left(#1\right)}}

\newcommand{\cnote}[2]{%
  \ifnotes
    {\marginpar{\color{#1}\scriptsize \sf \raggedright{#2}}}%
%    {\marginpar{\color{#1}\small\raggedright\parbox[t]{\columnwidth}{#2}}}
  \fi}

\newcommand{\citereferences}{
    \ifbool{draft_formatting}{}{\textsc{\textcolor{red}{CITE}}}
}
\newcommand{\autogen}{\ifbool{draft_formatting}{}{\textsc{Autogen}}}
\newcommand{\okay}{
    \ifbool{draft_formatting}{}{\pramod{Okay till here}}
}
\newcommand{\addcitations}{
    \ifbool{draft_formatting}{}{\textcolor{red}{[CITE]}}
}
\newcommand{\pramod}[1]{
    \ifbool{draft_formatting}{}{
        \vspace{0.3cm}
        \hrule{} \textcolor{red}{[\textsc{#1}]} \hrule{} \vspace{0.3cm}
    }
}

\newcommand{\rezaul}[1]{
    \ifbool{draft_formatting}{}{\textcolor{orange}{[\textsc{#1}]}}
}

\newcommand{\aaron}[1]{
    \ifbool{draft_formatting}{}{\textcolor{blue}{[\textsc{#1}]}}
}

\newcommand{\michael}[1]{
    \ifbool{draft_formatting}{}{\textcolor{blue}{[\textsc{#1}]}}
}

\definecolor{orange}{rgb}{1,0.3,0}
\newcommand{\rezaulnote}[1]{
    \ifbool{draft_formatting}{}{\cnote{red}{Rezaul: #1}}
}

\def\comment#1{\hfill{\color{gray}{$\left\{\textrm{{\em{#1}}}\right\}$}}}
\def\lcomment#1{\hfill{\color{gray}{$\left\{\textrm{{\em{#1}}}\right.$}}}
\def\rcomment#1{\hfill{\color{gray}{$\left.\textrm{{\em{#1}}}\right\}$}}}
\def\fcomment#1{\hfill{\color{gray}{$\textrm{{\em{#1}}}$}}}
\def\func#1{\textrm{\bf{\sc{#1}}}}
\def\funcbf#1{\textrm{\textbf{\textsc{#1}}}}
\def\F#1#2{\func{${#1}_{#2}$}}
\def\L#1#2{\func{${#1}_{loop\text{-}#2}$}}
\newcommand{\tim}[1]{\textbf{TIME: \{} #1 \textbf{\}}}
\newcommand{\oq}{\mbox{``}}
\newcommand{\cq}{\mbox{''}}
\newcommand{\bi}[1]{\textit{\textbf{#1}}}
%\newcommand{\vtxt}[2]{ \rotatebox{90}{ \begin{tabular}{p{#1}} \centering{#2} \end{tabular} } }
\newcommand{\vtxt}[2]{ \rotatebox{90}{ \centering{#2} } }

\newcommand{\vgap}{\vspace{-0.2cm}}
\newcommand{\m}[1]{\mathscr{#1}}
\newcommand{\M}{\textsf{M}}
\newcommand{\dbsp}{\textsf{D-BSP}}
\newcommand{\bfm}[1]{\mbox{\boldmath $#1$}}
\def\phead#1{~\\\noindent{\bf{#1.}}}
\newcommand{\embox}{\mbox{\hspace*{1em}}}
\newcommand{\para}[1]{\vspace{0.1cm}\noindent{\bf{#1.}}}
\newcommand{\vitem}{\vspace{-0.3cm}\item}
\newcommand{\codesize}{\small}
\newcommand{\codewidth}{0.98\textwidth}

\newcommand{\fnc}[1]{\textrm{\bf{\sc{#1}}}}
\newcommand{\TBL}[2]{\begin{tabular}{@{} #1 @{}} #2 \end{tabular}}
\newcommand{\ft}[1]{\footnotesize #1}
\newcommand{\T}{\hspace{1em}}

% math notations
\newcommand{\R}{\ensuremath{\mathbb R}}
\newcommand{\Z}{\ensuremath{\mathbb Z}}
\newcommand{\N}{\ensuremath{\mathbb N}}
%\newcommand{\F}{\ensuremath{\mathcal F}}
\newcommand{\SymGrp}{\ensuremath{\mathfrak S}}

\definecolor{gray}{rgb}{0.3,0.3,0.3}

\def\comment#1{\hfill{\color{gray}{$\left\{\textrm{{\em{#1}}}\right\}$}}}
\def\lcomment#1{\hfill{\color{gray}{$\left\{\textrm{{\em{#1}}}\right.$}}}
\def\rcomment#1{\hfill{\color{gray}{$\left.\textrm{{\em{#1}}}\right\}$}}}
\def\fcomment#1{\hfill{\color{gray}{$\textrm{{\em{#1}}}$}}}
\def\func#1{\textrm{\bf{\sc{#1}}}}
\def\funcbf#1{\textrm{\textbf{\textsc{#1}}}}


% asymptotic notations
\newcommand{\Oh}[1]{{\mathcal O}\left({#1}\right)}
\newcommand{\LOh}[1]{{\mathcal O}\left({#1}\right.}
\newcommand{\ROh}[1]{\left.{#1}\right)}
\newcommand{\oh}[1]{{o}\left({#1}\right)}
\newcommand{\Om}[1]{{\Omega}\left({#1}\right)}
\newcommand{\om}[1]{{\omega}\left({#1}\right)}
\newcommand{\Th}[1]{{\Theta}\left({#1}\right)}
\newcommand{\LTh}[1]{{\Theta}\left({#1}\right.}
\newcommand{\RTh}[1]{\left.{#1}\right)}

\newcommand{\xif}{{\bf{{if~}}}}
\newcommand{\xthen}{{\bf{{then~}}}}
\newcommand{\xelse}{{\bf{{else~}}}}
\newcommand{\xelseif}{{\bf{{elseif~}}}}
\newcommand{\xfi}{{\bf{{fi~}}}}
\newcommand{\xfor}{{\bf{{for~}}}}
\newcommand{\xeach}{{\bf{{each~}}}}
\newcommand{\xto}{{\bf{{to~}}}}
\newcommand{\xdownto}{{\bf{{downto~}}}}
\newcommand{\xby}{{\bf{{by~}}}}
\newcommand{\xdo}{{\bf{{do~}}}}
\newcommand{\xrof}{{\bf{{rof~}}}}
\newcommand{\xwhile}{{\bf{{while~}}}}
\newcommand{\xendwhile}{{\bf{{endwhile~}}}}
\newcommand{\xrepeat}{{\bf{{repeat~}}}}
\newcommand{\xuntil}{{\bf{{until~}}}}
\newcommand{\xand}{{\bf{{and~}}}}
\newcommand{\xor}{{\bf{{or~}}}}
\newcommand{\xerror}{{\bf{{error~}}}}
\newcommand{\xprint}{{\bf{{print~}}}}
\newcommand{\xreturn}{{\bf{{return~}}}}
\newcommand{\xreducer}{{\bf{{reducer~}}}}
\newcommand{\xpfor}{{\bf{{par for~}}}}
\newcommand{\xrfor}{{\bf{{reduce for~}}}}
\newcommand{\xnil}{\textrm{\sc{nil}}}
\newcommand{\xparallel}{{\bf{{parallel:~}}}}
\newcommand{\xparallelfor}{{\bf{{parallel for~}}}}
\newcommand{\xpar}{{\bf{{par:~}}}}
\newcommand{\xblank}{{$\phantom{mmmmm}$}}
\newcommand{\xblankpar}{{$\phantom{www:}$}}
\newcommand{\xblankparallel}{{$\phantom{mmmmm n}$}}
\newcommand{\xspawn}{{\bf{\em{spawn~}}}}
\newcommand{\xsync}{{\bf{\em{sync~}}}}
\newcommand{\xcomment}{\hfill $\rhd$ }
%\newcommand{\T}{\hspace{0.3cm}}

\newcommand{\mfunc}[1]{\mathcal{#1}}
\newcommand{\MM}[1]{\mfunc{MM}#1}
\newcommand{\idp}{$\mathcal{I}$-$\m{DP}$}
\newcommand{\lalgo}{$\mathcal{I}$}
\newcommand{\rdp}{$\mathcal{R}$-$\m{DP}$}
\newcommand{\calgo}{$\mathcal{R}$}
\newcommand{\fracdp}{$\m{FRACTAL}$-$\m{DP}$}

\newcommand{\nvs}{\vspace{-0.15cm}}
\newcommand{\vsitem}{\vspace{0cm}\item}

% to add new commands to algorithmic environment
\makeatletter
\newcommand{\ALOOP}[1]{\ALC@it\algorithmicloop\ #1%
  \begin{ALC@loop}}
\newcommand{\ENDALOOP}{\end{ALC@loop}\ALC@it\algorithmicendloop}
\newcommand{\algorithmicinput}{\textbf{Input:}}
\newcommand{\INPUT}{\item[\algorithmicinput]}
\newcommand{\algorithmicoutput}{\textbf{Output:}}
\newcommand{\OUTPUT}{\item[\algorithmicoutput]}
\newcommand{\algorithmicinvoke}{\textbf{Invoke:}}
\newcommand{\INVOKE}{\item[\algorithmicinvoke]}
\makeatother


\newcommand{\cmark}{\ding{51}}%
\newcommand{\xmark}{\ding{55}}%
\definecolor{desiredblue}{rgb}{0.803,0.940,0.995}
\definecolor{gray}{rgb}{0.3,0.3,0.3}
\colorlet{lightblue}{desiredblue}
\colorlet{lightred}{red!15}
\colorlet{lightgreen}{green!15}
\colorlet{mediumblue}{blue!25}
\colorlet{mediumred}{red!25}
\colorlet{mediumgreen}{green!25}
\colorlet{lightyellow}{yellow!15}
\colorlet{framecolor}{yellow!20}
\colorlet{frameloopcolor}{yellow!20}

\surroundwithmdframed[
  backgroundcolor   = algocolor,
  hidealllines      = true,
  innerleftmargin   = 0.5em,
  innerrightmargin  = 0.5em,
  innertopmargin    = -1.1em,
  innerbottommargin = 0em,
  leftline=true,
  rightline=true,
  bottomline=true,
  topline=true,
  skipabove         = .5\baselineskip,
  skipbelow         = .5\baselineskip
]{algorithm}

\newcommand{\parll}{\textbf{par:} }
\newcommand{\phantomparll}{\phantom{\textbf{par: } }}

\makeatletter
   \newcommand\figcaption{\def\@captype{figure}\caption}
   \newcommand\tabcaption{\def\@captype{table}\caption}
\makeatother

%--------------- TCOLORBOX -------------------

\colorlet{algotitlebarcolor}{green!20}
\colorlet{algotitlebarcolortop}{lightblue}
\colorlet{algotitlebarcolorbottom}{lightblue}
\colorlet{framecolor}{yellow}
\colorlet{framecolortop}{white}
\colorlet{framecolorbottom}{white}
\colorlet{bluetop}{cyan!40}
\colorlet{bluebottom}{cyan!20}
%\colorlet{redtop}{RubineRed!40}
%\colorlet{redbottom}{RubineRed!20}
\colorlet{greentop}{white!40}
\colorlet{greenbottom}{white!20}
\colorlet{shadowcolor}{gray}
\colorlet{algotitlecolor}{black}
\colorlet{algoframecolor}{gray}
\colorlet{tabletitlecolor}{desiredblue}
%\colorlet{tablefillcolor}{Goldenrod!40}
%\colorlet{tablefillcolor2}{Goldenrod!20}
\colorlet{algocolor}{black}

\newtcolorbox{mycolorbox}[1]
{skin=enhanced,colbacktitle=algotitlebarcolor,coltitle=algotitlecolor,colback=algocolor,colframe=algoframecolor,boxrule=1pt,left=5mm,right=1mm,title style={top color=algotitlebarcolortop, bottom color=algotitlebarcolorbottom},interior style={top color=framecolortop, bottom color=framecolorbottom},title={\codesize #1}}

\newcommand{\algotopspace}{\vspace{-0.25cm}}
\newcommand{\algomiddlespace}{\vspace{0.1cm}}
\newcommand{\algobottomspace}{\vspace{-0.15cm}}
\newcommand{\algodiffspace}{\vspace{-0.15cm}}
\newcommand{\algoinput}{{\bf{{Input:~}}}}
\newcommand{\algooutput}{{\\\bf{{Output:~}}}}
\newcommand{\algorequire}{{\bf{{Require:~}}}}

%--------------- TABLE COLOR ---------------
\arrayrulecolor{gray}
\setlength{\arrayrulewidth}{0.4mm}
\renewcommand{\arraystretch}{0.8}
\newsavebox{\tablebox}
\NewEnviron{colortabular}[1]{%
  \addtolength{\extrarowheight}{1ex}%
  %\rowcolors{2}{tablefillcolor}{tablefillcolor2}
  \savebox{\tablebox}{%
  %\scalebox{0.85}{
    \begin{tabular}{#1}%
	  \rowcolor{tabletitlecolor}
      \BODY%
    \end{tabular}}
	%}
  \begin{tikzpicture}
    \begin{scope}
      \clip[rounded corners=1ex] (0,-\dp\tablebox) -- (\wd\tablebox,-\dp\tablebox) -- (\wd\tablebox,\ht\tablebox) -- (0,\ht\tablebox) -- cycle;
      \node at (0,-\dp\tablebox) [anchor=south west,inner sep=0pt]{\usebox{\tablebox}};
    \end{scope}
    \draw[rounded corners=1ex] (0,-\dp\tablebox) -- (\wd\tablebox,-\dp\tablebox) -- (\wd\tablebox,\ht\tablebox) -- (0,\ht\tablebox) -- cycle;
  \end{tikzpicture}
}

\setlist[enumerate,1]{leftmargin=\dimexpr 26pt-0.3cm}

