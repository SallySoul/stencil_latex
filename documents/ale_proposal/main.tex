\documentclass{article}
% Setup for bibliography
\usepackage[
backend=biber,
style=numeric-comp,
]{biblatex}
\addbibresource{../../references.bib}

% Pretty self explanatory
% We sue this in title bits
\usepackage{datetime}

% Standard math packages / setup
\usepackage{amsmath} 
\usepackage{amsfonts}
\usepackage{amsthm}
\usepackage{amssymb} 
\usepackage{accents}
\usepackage{mathrsfs}
\usepackage{mathtools}

\usepackage{bm}

%\newtheorem{lemma}{Lemma}
%\newtheorem{theorem}{Theorem}
%\newtheorem{definition}{Definition}

% So we can import pngs
\usepackage{graphicx} 

% This gives us nice clickable links 
% https://www.overleaf.com/learn/latex/Hyperlinks#Styles_and_colours
\usepackage{hyperref}
\hypersetup{
    colorlinks=true,
    linkcolor=blue,
    citecolor=blue,
    filecolor=magenta,      
    urlcolor=cyan,
    pdftitle={Monte Carlo Methods (DRAFT)},
    pdfpagemode=FullScreen,
    }
\urlstyle{same}

% Allows us to define colors
% We use this in the next block, listings
\usepackage{color}
\definecolor{dkgreen}{rgb}{0,0.6,0}
\definecolor{gray}{rgb}{0.5,0.5,0.5}
\definecolor{mauve}{rgb}{0.58,0,0.82}

% Allows us to include 
\usepackage{listings}
\lstset{frame=tb,
  language={},
  aboveskip=3mm,
  belowskip=3mm,
  showstringspaces=false,
  columns=flexible,
  basicstyle={\small\ttfamily},
  numbers=none,
  numberstyle=\tiny\color{gray},
  keywordstyle=\color{blue},
  commentstyle=\color{dkgreen},
  stringstyle=\color{mauve},
  breaklines=true,
  breakatwhitespace=true,
  tabsize=4
}

% Adds bulletized outlines with outline environment
\usepackage{outlines}

% Tikz
\usepackage{tikz}

% Colors
\usepackage{xcolor}
\definecolor{uconnblue}{rgb}{0.08, 0.18, 0.28}
\definecolor{intactblue}{rgb}{0.13, 0.26, 0.45}
\definecolor{mastercamred}{rgb}{0.83, 0.01, 0.23}


\usepackage[
  letterpaper,
  left=2cm,
  right=2cm,
  top=2cm,
  bottom=2cm
]{geometry}

\setlength{\parindent}{20pt}

\title{Code Generation for CFD on Moving Voronoi Meshes}
\author{Russell Bentley}

\begin{document}

\twocolumn

\maketitle

\section{Introduction}

Computational Fluid Dynamics (CFD) is a computational black hole
that has lead to a proliferation of techniques that maximize the 
efficieny of available resources by tailoring the simulation to the ``regime.''
Simulation for science and engineering is 
increasingly looking to handle complex multi-scale and multi-regime 
fluid dynamics problems that may interact with other dyanamics while 
simultaneously providing numerical accuracy gaurentees.
Astrophysics in particular demands a large number of different ``physics'' 
interacting with fluid dynamics and large and dynamic scale, 
while needing a higher degree of Galilean Invariance.

Arbitrary-Lagrangain-Eulierian (ALE) methods 
is a broad range of techniques for solving conservation laws
(hyperbolic equations)



seek to combine
the 
Here our mesh can move in arbitrary ways, often informed by both fluid velocity and
mesh quality considerations.
Moving Voronoi meshes re-mesh from a set of generating points at each time step,
which both
side steps the issue of tangled elements after topology changes,
and allows for easy spatial refinement.

AREPO solves magneto-hydrodynamics, and is the main implementation for this work.
The code base follows in the lineage of \textsc{Gadget-2} and \textsc{Gadget} which 
were both smoothed particle hydro-dynamics (SPH) based codes.
AREPO's architecture has simultaneously limited the application of ALE + Moving Voronoi Mesh
to other domains like Aerospace, and has hindered the addition of new features for astrophysiscists 
to expand its modelling capabilities.


Research from other numerical techniques shows us that modern architectures can simultaneously
improve performance of a numerical scheme and expand the ability extend and explore.
Projects like FeNIcs for FEA and lbmpy + waLBerla show us that suitable high-level language and a code generation 
framework can bring many benefits.
Separate the ``business logic'' from optimizations.
pyStencil is a core project of interest and could likely be used directly or in part when 
building our framework.

\section{Outcomes}
\begin{outline}
  \1 Moving Voronoi Mesh Runtime \2 Handle Domain Partitioning
  \2 Checkpointing
  \2 Meshing and tree construction
  \2 2D / 3D
  \2 Extend to space-time elements in the future
  \1 Code Generation Framework
  \2 High Level description of 
  \3 Governing equations (hyperbolic conservations laws + source terms)
  \3 Conserved Quantities
  \2 Generation of core CFD solvers
  \1 Example Applications
  \2 Hypersonics 
  \3 Compressible fluid + Boundary Conditions
  \3 Shockwaves!
  \2 CMZ Study
  \3 MHD
  \3 Self Gravity
  \3 Analytical Gravity Term
  \3 Star Particles / Agents
  \3 Chemistry Network
  \2 FZ-CO4 (or w/e)
  \3 Very advanced Conservation system, solve for gravity locally!
\end{outline}

\printbibliography

\end{document}


