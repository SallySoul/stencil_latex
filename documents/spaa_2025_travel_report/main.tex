\documentclass{article}
% Setup for bibliography
\usepackage[
backend=biber,
style=numeric-comp,
]{biblatex}
\addbibresource{../../references.bib}

% Pretty self explanatory
% We sue this in title bits
\usepackage{datetime}

% Standard math packages / setup
\usepackage{amsmath} 
\usepackage{amsfonts}
\usepackage{amsthm}
\usepackage{amssymb} 
\usepackage{accents}
\usepackage{mathrsfs}
\usepackage{mathtools}

\usepackage{bm}

%\newtheorem{lemma}{Lemma}
%\newtheorem{theorem}{Theorem}
%\newtheorem{definition}{Definition}

% So we can import pngs
\usepackage{graphicx} 

% This gives us nice clickable links 
% https://www.overleaf.com/learn/latex/Hyperlinks#Styles_and_colours
\usepackage{hyperref}
\hypersetup{
    colorlinks=true,
    linkcolor=blue,
    citecolor=blue,
    filecolor=magenta,      
    urlcolor=cyan,
    pdftitle={Monte Carlo Methods (DRAFT)},
    pdfpagemode=FullScreen,
    }
\urlstyle{same}

% Allows us to define colors
% We use this in the next block, listings
\usepackage{color}
\definecolor{dkgreen}{rgb}{0,0.6,0}
\definecolor{gray}{rgb}{0.5,0.5,0.5}
\definecolor{mauve}{rgb}{0.58,0,0.82}

% Allows us to include 
\usepackage{listings}
\lstset{frame=tb,
  language={},
  aboveskip=3mm,
  belowskip=3mm,
  showstringspaces=false,
  columns=flexible,
  basicstyle={\small\ttfamily},
  numbers=none,
  numberstyle=\tiny\color{gray},
  keywordstyle=\color{blue},
  commentstyle=\color{dkgreen},
  stringstyle=\color{mauve},
  breaklines=true,
  breakatwhitespace=true,
  tabsize=4
}

% Adds bulletized outlines with outline environment
\usepackage{outlines}

% Tikz
\usepackage{tikz}

% Colors
\usepackage{xcolor}
\definecolor{uconnblue}{rgb}{0.08, 0.18, 0.28}
\definecolor{intactblue}{rgb}{0.13, 0.26, 0.45}
\definecolor{mastercamred}{rgb}{0.83, 0.01, 0.23}


\usepackage[
  letterpaper,
  left=2cm,
  right=2cm,
  top=2cm,
  bottom=2cm
]{geometry}

\setlength{\parindent}{20pt}

\title{SPAA 2025 Travel Report}
\author{Russell Bentley}

\begin{document}

\twocolumn

\maketitle
\section{Introduction}

SPAA 2025 represented several important milestones in my academic career.
It was my first academic conference,
as well as the first time I presented at any conference.
It was also the culmination of my first year as a PhD student.
Having a paper published in the conference was a significant 
recognition of our work. 
I am grateful for the funding provided by SPAA and NSF 
that helped make this trip possible for me.

\section{Presenting}

I attended SPAA 2025 as a presentor for our paper 
``\textit{Applying Fast Fourier Transforms to Accelerate Spatially and Temporally Inhomogeneous Stencil Computations}.''
Preparing for the presentation weighed heavy on me over the summer.
I was worried that I would have to present first, 
preventing observations of norms and frantic last minute adjustments.
As it happened, I presented second so this worry did come to pass.

I iterated on the talk over the summer,
getting helpful feedback from my advisor and other members of our lab.
However, the night prior to the talk a more experienced colleague 
from another lab kindly helped me with my presentation.
We simplified things and tightened up the core narrative.
Although the last minute changes left me under practiced, 
I think the talk went quite well.

I only got a couple questions in the Q\&A for the talk, 
but I did have a number of folks approach me about it later in the day.
I made some fantastic connections through these conversations!

Presenting and watching talks at SPAA taught me alot about the format.
Certainly $18$ minutes is a short time frame. 
The more important constraint I hadn't understood before was the limited audience attention.
Folks are sitting through a large number of talks 
and can't be expected to understand the intricacies of every paper.
The best talks were hyper focused on explaining one new concept.
The talks I didn't like as much were more like a typical lecture 
in that they aimed to comprehensively explain the entire paper.

In total, any future conference presentations I give
will be better for the experience I gained at SPAA 2025.

\section{Conference Highlights}

SPAA 2025 was an intense experience, with hours of talks every day and lots of socializing.

My favorite workshop was the first one, ``\textit{Concurrent Data Structures in RDMA}.''
I have worked professionally on HPC software, and it is a topic that I would like to continue to focus on.
I've been curious to learn more about RDMA for a long time.
The hands on workshop really opened my eyes to how RDMA works in practice.
In particular the implications for data structure design have been on my mind ever since.

Although there were many great talks and keynotes, a standout for me was the presentation for 
``\textit{Decoupled Fallback: A Portable Single-Pass GPU Scan}.'' 
The WebGPU standard is of interest to me, and is something I've used as a target for projects in the past.
I place a high value on not getting locked to one vendor, but there are some substantial tradeoffs for this.
This paper was exciting because it demonstrated a state-of-the-art implementation of a fundamental GPU algorithm 
that was built to target this vendor agnostic standard. 
I found that re-assuring and exciting!

I met numerous members of the parallel algorithm research community at SPAA.
The highlight for me was meeting my advisor's advisor, Dr. Vijaya Ramachandran.
Our conversation was helpful to me in contextualizing both my advisor's work and my own within the broader community.

\section{Conclusion}

SPAA 2025 was a new an exciting experience for me.
The process of publishing and presenting a paper was rewarding of course.
It was also invigorating to be exposed to the state-of-the-art 
in parallel algorithms research.
I am excited to grow into this community and hopefully continue to contribute over the coming years.


\end{document}

