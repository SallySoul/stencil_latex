\documentclass{article}
% Setup for bibliography
\usepackage[
backend=biber,
style=numeric-comp,
]{biblatex}
\addbibresource{../../references.bib}
\addbibresource{../../website_references.bib}


% Pretty self explanatory
% We sue this in title bits
\usepackage{datetime}

% Standard math packages / setup
\usepackage{amsmath} 
\usepackage{amsfonts}
\usepackage{amsthm}
\usepackage{amssymb} 
\usepackage{accents}
\usepackage{mathrsfs}
\usepackage{mathtools}

\usepackage{bm}

%\newtheorem{lemma}{Lemma}
%\newtheorem{theorem}{Theorem}
%\newtheorem{definition}{Definition}

% So we can import pngs
\usepackage{graphicx} 

% This gives us nice clickable links 
% https://www.overleaf.com/learn/latex/Hyperlinks#Styles_and_colours
\usepackage{hyperref}
\hypersetup{
    colorlinks=true,
    linkcolor=blue,
    citecolor=blue,
    filecolor=magenta,      
    urlcolor=cyan,
    pdftitle={Monte Carlo Methods (DRAFT)},
    pdfpagemode=FullScreen,
    }
\urlstyle{same}

% Allows us to define colors
% We use this in the next block, listings
\usepackage{color}
\definecolor{dkgreen}{rgb}{0,0.6,0}
\definecolor{gray}{rgb}{0.5,0.5,0.5}
\definecolor{mauve}{rgb}{0.58,0,0.82}

% Allows us to include 
\usepackage{listings}
\lstset{frame=tb,
  language={},
  aboveskip=3mm,
  belowskip=3mm,
  showstringspaces=false,
  columns=flexible,
  basicstyle={\small\ttfamily},
  numbers=none,
  numberstyle=\tiny\color{gray},
  keywordstyle=\color{blue},
  commentstyle=\color{dkgreen},
  stringstyle=\color{mauve},
  breaklines=true,
  breakatwhitespace=true,
  tabsize=4
}

% Adds bulletized outlines with outline environment
\usepackage{outlines}

% Tikz
\usepackage{tikz}

% Colors
\usepackage{xcolor}
\definecolor{uconnblue}{rgb}{0.08, 0.18, 0.28}
\definecolor{intactblue}{rgb}{0.13, 0.26, 0.45}
\definecolor{mastercamred}{rgb}{0.83, 0.01, 0.23}


\usepackage[
  letterpaper,
  left=2cm,
  right=2cm,
  top=2cm,
  bottom=2cm
]{geometry}

\setlength{\parindent}{20pt}

\title{CSE 642: Scribe Notes \\ September 5, 2025}
\author{Russell Bentley}


\begin{document}
\maketitle

\section{Introduction}

Quick discussion of problems to discuss. 
\begin{itemize}
\item We could resume conversation on polygon crosssections.
\item Rezaul has some problems that are ready to discuss.
\item Joe had a problem or two that are ready, including ``Connections.''
\end{itemize}

\section{Lucas's Polygon Problem}

\subsection{Review the Problem Statement}

Given a set $S$ of $n$ points on the plane, and a set $L$ of $m$ lines in the plane, all in general positions.
Each line has a n integer label that is non-negative and even.
Question one: Does there exist a simple polygon $P$ whose vertices are $S$ and that cross each line the number of times given by its label.
Question two: What is the complexity of deciding this and constructing the polygon if it exists.

(See poly_example.png)

\subsection{Review of previous discussion}

Some discussion from last time:
* Are we allowed to introduce new points?
* It seems that if we can find a solution with some of the points then there will be a solution that uses all the points (Zoran), but perhaps not.
* 

\subsection{New Discussion}

Lucas found a new paper that discusses this problem ``\href{https://www.sciencedirect.com/science/article/abs/pii/0020019089900124}{Reconstructing graphs from cut-set sizes}.''

For this work:
Unknown graph straight line graph with $n$ vertices, points in general position.
In this setting we can ask cut queries, specified by a line $L$ in the plane and returns how many edges are crossed.
The Goal is to obtain a strategy to discover $G$ completely using as few cuts as possible.

In the paper, $\Oh(n^2)$ queries are sufficient and sometimes necessary in general.
Just query every 
Our case, we know that $G$ is a cycle.

(Grab fig fro paper that shows necessary probes)
We can simulate an edge query with four cut queries.
So (n choose 2) * 4 queries.

Question: how tight is the constant in this complexity bound?

Our problem, the points $S$, the lines $L$, and the answers to the crossing number queries are given. 
We are to decide if there exists a cycle graph $G$ on $S$ with no crossing edges.

Another question: Are positive answers unique?
Lets come up with a small counter example.
Well trivially, if no lines are given, or all lines are zero and all points $S$ lie in a common face, 
with the points not in convex position, then there are multiple polygons satisfying the constraints.

Another way to connect these two problems is to decide of the oracle is lying or not. 
That is, a motivation for Lucas's problem is the catch a lying oracle.

So does knowing something about the topology of what we're looking for (simple polygon, embedded tree, ect) help at all?
This was an open question from the paper.
We should look up scribe notes from when this was discussed in the seminar last time (Early October 2021?)

Rezaul: Given 2 polygons $P$, $Q$ can they be translated / rotated so that they have $k$ crossings?
For pure translation, standard 2D CG methods apply (good HW).
With rotations (possibly scaling, etc) more general methods apply.

Related graph drawing problem: Given $2$ fixed graphs $G_1$ and $G_2$, translate / rotate / scale (with limitations) one with respect to the other
in order to minimize (or max?) the number of crossings. 

We're talking around many interesting variants, but haven't discussed the original problem!

Steiner points / version?

Return to problem:
a) If we are allowed any number of Steiner points, wherever we want them, the problem is easy.
(Last time we asked if $S$ are empty, but even if $S$ is non-empty it is easy).

Claim 2:
If there is a not a red cell that is intersected by all blue lines, then it is hopeless, no polygon exists (no matter the set $S$).


\end{document}







